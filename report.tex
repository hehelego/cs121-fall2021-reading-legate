\documentclass{article}
\usepackage{indentfirst}
\usepackage{graphicx}
\usepackage{algorithmicx}
\usepackage[outputdir=tex-output, cachedir=tex-output]{minted}
\usepackage[a4paper, total={170mm,257mm}, left=20mm, top=20mm]{geometry}

\usepackage{hyperref}
\hypersetup{
    colorlinks=true,
    linkcolor=blue,
    filecolor=magenta,      
    urlcolor=cyan,
    pdftitle={Overleaf Example},
    pdfpagemode=FullScreen,
}
\urlstyle{same}


\usepackage[backend=biber]{biblatex}
\addbibresource{ref.bib}

\usepackage{fancyhdr}
\pagestyle{fancy}
\fancyhf{}
\newenvironment{newSec}[1]{
	\section{#1}
	\lhead{#1}
}{ \newpage }
\newenvironment{newSecNoNP}[1]{
	\section{#1}
	\lhead{#1}
}{}
\newenvironment{newSubsec}[1]{
	\subsection{#1}
}{}
\lfoot{CS121 Reading Project}
\cfoot{Review on Accelerated NumPy}
\rfoot{Page \thepage}
\renewcommand{\headrulewidth}{1pt}
\renewcommand{\footrulewidth}{1pt}
\setlength{\headsep}{10pt}



\begin{document}
\title{Accelerating {NumPy}\cite{numpy} for Data Science: A Summary on Legate}
\author{Cheng Peng\footnote{2020533068\quad \href{mailto:pengcheng2@shanghaitech.edu.cn}{pengcheng2@shanghaitech.edu.cn}}}
\maketitle
\tableofcontents
\newpage

\begin{newSecNoNP}{Abstract}
	Python is a widely used general programming language that is known for the simple and clear syntax.
	With the support of huge standard library and third-part packages, a enormous python eco-system has formed in the past two decade.\par
	It is popular in the filed of scientific computation and data science, however python is not designed for high performance computing.
	The object memory model and the interpreter design made it hard for python program to be fast and scalable.
	To overcome this barriers, plenty of research projects have arised.
	Some tries to address this issues by providing alternative runtime for python,
	while others re-implement certain python libraries with high performance code and link the external code with FFI.\par
	In this paper review project, we will dive into legate numpy, a python library aims at accelerating python with distributed GPU cluster.
\end{newSecNoNP}

\begin{newSec}{Background}
	numpy\cite{numpy}
	cupy\cite{cupy}
\end{newSec}

\begin{newSec}{The Legate NumPy System}
	{legate numpy}\cite{legate}
	\begin{newSubsec}{overview}
	\end{newSubsec}

	\begin{newSubsec}{the legion programming model}
	\end{newSubsec}

	\begin{newSubsec}{the data model}
	\end{newSubsec}
	\begin{newSubsec}{the task model}
	\end{newSubsec}
	\begin{newSubsec}{control replication}
	\end{newSubsec}
	\begin{newSubsec}{heuristical optimizations}
	\end{newSubsec}

	\begin{newSubsec}{benchmarks}
	\end{newSubsec}
\end{newSec}

\begin{newSec}{Related Works}
	As we have mentioned in the introductory section, people have been trying to accelerate python ever since the rise of data science.
	In this section, we are to cover a few more python libraries/frameworks/runtimes/compilers for speeding up large scale data science applications.\par
	Only general purpose open source project are covered.
	In fact, a variety of domain-specified numpy implementations do exists and a considerable number of enterprises have developed their own closed-source replacement of numpy.
	\begin{newSubsec}{Theano / Aesara}
		Aiming at accelerating mathematical expressions, Theano\cite{theano}, consisting of a set of numpy-like API and a compiler.
		Before the python code is executed, Theano compiler transform it into optimized high performance C++ code and link it as a dynamically loaded python modules.\par
		Theano is the pioneer of differentiable programming paradigm, the auto symbolic differentiation features
		allow programmers to implement numerical optimization algorithm such as the gradient descent without manually evaluating the derivatives, which is painful and error-prone.
		This key feature not only alleviates manual symbolic calculation, but also opens a door for advanced code generation.
		Having full access to the computational graph, Theano applies local graph transformations to eliminate unnecessary, slow or numerically unstable expression patterns.\par
		Theano is orginally a research project from Google MILA groups, however MILA stopped the development and maintance in 2017.
		It is being continued as aesara, a community fork of theano.
	\end{newSubsec}

	\begin{newSubsec}{Numba}
		Numba is similar to Theano, both libraries try to compile python code into high performance. When comparing them, Numba is more powerful and flexible.\par
		To overcome the performance degrade of python interpreter and take the advantage of parallel execution,
		the only modification needed is simply marking functions with \mintinline{python}{@jit,@cuda.jit,@cfunc} decorator.
		The code then get transformed into optimized high performance code and can even utilize common hardware accelerators.\par
		Numba implement a JIT (just in time) compiler to transform normal python code into LLVM IR.
		They do optimization on the LLVM IR level and generate machine code from it.
		The benefit of using LLVM IR is that Numba can support for targeting different hardwares such as multi-core CPUs, GPGPUs and Google TPUs.
	\end{newSubsec}


	\begin{newSubsec}{CuPy: array programming on a single NVIDIA GPU}
		Prior to the birth of legate, the researchers at NVIDIA developed a drop-in replacement for numpy called CuPy\cite{cupy}.
		It is highly compatible with the N-dimensional array API provided by NumPy.
		Typically, to turn a CPU code into CUDA program can be achieve by replacing \mintinline{python}{import numpy as np} with \mintinline{python}{import cupy as np}.\par
		CuPy does not simple move the data onto GPU main memory and implement array operations with CUDA kernels.
		It is linked with NVIDIA's CUDA toolkit. cuBLAS, cuSOLVE, cuRAND, cuFFT are integrated into CuPy.
		Cupy also include support low-level CUDA codes. User-defined CUDA kernels written in C++ are wrapped and loaded readily.
	\end{newSubsec}

	\begin{newSubsec}{Dask}
		The previous three libraries focus on single node performance boost,
		while Dask\cite{dask} is targeted fully utilizing the power of multi-node cluster.
		Dask is a task-based runtime system for parallel and distributed computation.\par
		\mintinline{python}{dask.array}, \mintinline{python}{dask.bag} and \mintinline{python}{dask.dataframe} are the three main data structure provided,
		programmer use the then to compose programs.
		Dask analysis the program and divides the computation into tasks.
		The tasks are organized in a DAG called dask graph. The edges and vertices represent dependencies and computation tasks respectively.
		Dask explore the graph and schedule execution of tasks dynamically until completion.\par
		Dask provides richer data structures than legate, however the performance is lower than legate.
		In the legate paper, researchers at NVIDIA compared the performance and scalability of dask and legate.
		They concluded that legate outperforms dask in the sense of absolute speedup and weak-scalability.
	\end{newSubsec}

	\begin{newSubsec}{Tensorflow, PyTorch and MXNet}
		The last five years have witnessed the tide of deep learning.
		As the typical size of neural networks grows, training and inference require more computation power and memory foodprint.\par
		Tensorflow, PyTorch and MXNet are three popular deep learning frameworks, each has implemented tensor (which is a fancier name for ndarray) facilities on GPU.
	\end{newSubsec}
\end{newSec}

\begin{newSecNoNP}{Conclusion}
	In this paper review project, we give an anatomy of the legate system, a NVIDIA research project provides accelerated numpy-compatible N-dimensional array operations on distributed GPU cluster.\par
	In the beginning we give a overview of the scientific python ecosystem, where numpy plays a central role.
	We then explored the legion runtime upon which the legate is built.
	Shortly after that, we presented a detailed explanation on how legate maps numpy data structures and computational task into legion data and tasks.\par
	In the last part, we covered several related works.
	We give a short summary on their mechanism and compared them with legate.
\end{newSecNoNP}
\printbibliography
\appendix

\begin{newSec}{Sample code}
	\begin{newSubsec}{NumPy}
		It is convenient to express matrix operations in NumPy API.
		Here we illustrate how to implement Jacobi iteration with NumPy.
		\inputminted{python}{src/numpy.py}
	\end{newSubsec}
	\begin{newSubsec}{Theano/Aesara}
		Here we demonstrate how to implement \href{https://aesara.readthedocs.io/en/latest/tutorial/examples.html#a-real-example-logistic-regression}{logistic regression} in aesara.
		\inputminted{python}{src/theano.py}
	\end{newSubsec}
	\begin{newSubsec}{Numba}
		The following code gives a simple example on numba's JIT decorator.
		\inputminted{python}{src/numba.py}
	\end{newSubsec}
	\begin{newSubsec}{CuPy}
		With CuPy, one can speedup common scientific computation task with NVIDIA CUDA toolkit.
		We demonstrate how to use cuFFT.
		\inputminted{python}{src/cupy.py}
	\end{newSubsec}
\end{newSec}


\end{document}
